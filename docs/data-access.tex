\documentclass[listings]{labreport}
\departmentsubject{Кафедра вычислительной техники}{Программирование интернет-приложений}
\titleparts{Курсовая работа, 2 этап}{Вариант 12831}
\students{Калугина Марина, Каюков Иван}

\begin{document}

\maketitlepage

\section*{Цель работы}

На данном этапе необходимо реализовать уровень \textit{DataAccess} 
разрабатываемой системы. Уровень \textit{DataAccess} должен содержать 
\textit{CRUD API} ко всем сущностям 
базы данных, а также ряд дополнительных функций, номенклатура которых 
определяется конкретной предметной областью, согласуется с преподавателем 
и фиксируется в техническом задании.

\section*{Исходный код}

Исходный код доступен по адресу 
\texttt{https://github.com/band-of-four/RoboGit}.

\section*{Вывод}

На данном этапе мы разработали \textit{CRUD API} ко всем сущностям базы данных,
и, также, некоторые функции которые могут нам понадобиться для реализации
бизнес-логики.

\end{document}
